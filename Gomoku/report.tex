\documentclass[conference,compsoc]{IEEEtran}

% *** CITATION PACKAGES ***
%
\ifCLASSOPTIONcompsoc
  % IEEE Computer Society needs nocompress option
  % requires cite.sty v4.0 or later (November 2003)
  \usepackage[nocompress]{cite}
\else
  % normal IEEE
  \usepackage{cite}
\fi
% cite.sty was written by Donald Arseneau
% V1.6 and later of IEEEtran pre-defines the format of the cite.sty package
% \cite{} output to follow that of the IEEE. Loading the cite package will
% result in citation numbers being automatically sorted and properly
% "compressed/ranged". e.g., [1], [9], [2], [7], [5], [6] without using
% cite.sty will become [1], [2], [5]--[7], [9] using cite.sty. cite.sty's
% \cite will automatically add leading space, if needed. Use cite.sty's
% noadjust option (cite.sty V3.8 and later) if you want to turn this off
% such as if a citation ever needs to be enclosed in parenthesis.
% cite.sty is already installed on most LaTeX systems. Be sure and use
% version 5.0 (2009-03-20) and later if using hyperref.sty.
% The latest version can be obtained at:
% http://www.ctan.org/pkg/cite
% The documentation is contained in the cite.sty file itself.
%
% Note that some packages require special options to format as the Computer
% Society requires. In particular, Computer Society  papers do not use
% compressed citation ranges as is done in typical IEEE papers
% (e.g., [1]-[4]). Instead, they list every citation separately in order
% (e.g., [1], [2], [3], [4]). To get the latter we need to load the cite
% package with the nocompress option which is supported by cite.sty v4.0
% and later.





% *** GRAPHICS RELATED PACKAGES ***
%
\ifCLASSINFOpdf
  % \usepackage[pdftex]{graphicx}
  % declare the path(s) where your graphic files are
  % \graphicspath{{../pdf/}{../jpeg/}}
  % and their extensions so you won't have to specify these with
  % every instance of \includegraphics
  % \DeclareGraphicsExtensions{.pdf,.jpeg,.png}
\else
  % or other class option (dvipsone, dvipdf, if not using dvips). graphicx
  % will default to the driver specified in the system graphics.cfg if no
  % driver is specified.
  % \usepackage[dvips]{graphicx}
  % declare the path(s) where your graphic files are
  % \graphicspath{{../eps/}}
  % and their extensions so you won't have to specify these with
  % every instance of \includegraphics
  % \DeclareGraphicsExtensions{.eps}
\fi
% graphicx was written by David Carlisle and Sebastian Rahtz. It is
% required if you want graphics, photos, etc. graphicx.sty is already
% installed on most LaTeX systems. The latest version and documentation
% can be obtained at: 
% http://www.ctan.org/pkg/graphicx
% Another good source of documentation is "Using Imported Graphics in
% LaTeX2e" by Keith Reckdahl which can be found at:
% http://www.ctan.org/pkg/epslatex
%
% latex, and pdflatex in dvi mode, support graphics in encapsulated
% postscript (.eps) format. pdflatex in pdf mode supports graphics
% in .pdf, .jpeg, .png and .mps (metapost) formats. Users should ensure
% that all non-photo figures use a vector format (.eps, .pdf, .mps) and
% not a bitmapped formats (.jpeg, .png). The IEEE frowns on bitmapped formats
% which can result in "jaggedy"/blurry rendering of lines and letters as
% well as large increases in file sizes.
%
% You can find documentation about the pdfTeX application at:
% http://www.tug.org/applications/pdftex





% *** MATH PACKAGES ***
%
\usepackage{amsmath}
% A popular package from the American Mathematical Society that provides
% many useful and powerful commands for dealing with mathematics.
%
% Note that the amsmath package sets \interdisplaylinepenalty to 10000
% thus preventing page breaks from occurring within multiline equations. Use:
%\interdisplaylinepenalty=2500
% after loading amsmath to restore such page breaks as IEEEtran.cls normally
% does. amsmath.sty is already installed on most LaTeX systems. The latest
% version and documentation can be obtained at:
% http://www.ctan.org/pkg/amsmath





% *** SPECIALIZED LIST PACKAGES ***
%
\usepackage{algorithm}
\usepackage{algorithmic}
\renewcommand{\algorithmicrequire}{\textbf{Input:}}
\renewcommand{\algorithmicensure}{\textbf{Output:}}
% algorithmic.sty was written by Peter Williams and Rogerio Brito.
% This package provides an algorithmic environment fo describing algorithms.
% You can use the algorithmic environment in-text or within a figure
% environment to provide for a floating algorithm. Do NOT use the algorithm
% floating environment provided by algorithm.sty (by the same authors) or
% algorithm2e.sty (by Christophe Fiorio) as the IEEE does not use dedicated
% algorithm float types and packages that provide these will not provide
% correct IEEE style captions. The latest version and documentation of
% algorithmic.sty can be obtained at:
% http://www.ctan.org/pkg/algorithms
% Also of interest may be the (relatively newer and more customizable)
% algorithmicx.sty package by Szasz Janos:
% http://www.ctan.org/pkg/algorithmicx






% *** ALIGNMENT PACKAGES ***
%
\usepackage{array}
% Frank Mittelbach's and David Carlisle's array.sty patches and improves
% the standard LaTeX2e array and tabular environments to provide better
% appearance and additional user controls. As the default LaTeX2e table
% generation code is lacking to the point of almost being broken with
% respect to the quality of the end results, all users are strongly
% advised to use an enhanced (at the very least that provided by array.sty)
% set of table tools. array.sty is already installed on most systems. The
% latest version and documentation can be obtained at:
% http://www.ctan.org/pkg/array


% IEEEtran contains the IEEEeqnarray family of commands that can be used to
% generate multiline equations as well as matrices, tables, etc., of high
% quality.




% *** SUBFIGURE PACKAGES ***
%\ifCLASSOPTIONcompsoc
%  \usepackage[caption=false,font=footnotesize,labelfont=sf,textfont=sf]{subfig}
%\else
%  \usepackage[caption=false,font=footnotesize]{subfig}
%\fi
% subfig.sty, written by Steven Douglas Cochran, is the modern replacement
% for subfigure.sty, the latter of which is no longer maintained and is
% incompatible with some LaTeX packages including fixltx2e. However,
% subfig.sty requires and automatically loads Axel Sommerfeldt's caption.sty
% which will override IEEEtran.cls' handling of captions and this will result
% in non-IEEE style figure/table captions. To prevent this problem, be sure
% and invoke subfig.sty's "caption=false" package option (available since
% subfig.sty version 1.3, 2005/06/28) as this is will preserve IEEEtran.cls
% handling of captions.
% Note that the Computer Society format requires a sans serif font rather
% than the serif font used in traditional IEEE formatting and thus the need
% to invoke different subfig.sty package options depending on whether
% compsoc mode has been enabled.
%
% The latest version and documentation of subfig.sty can be obtained at:
% http://www.ctan.org/pkg/subfig




% *** FLOAT PACKAGES ***
%
%\usepackage{fixltx2e}
% fixltx2e, the successor to the earlier fix2col.sty, was written by
% Frank Mittelbach and David Carlisle. This package corrects a few problems
% in the LaTeX2e kernel, the most notable of which is that in current
% LaTeX2e releases, the ordering of single and double column floats is not
% guaranteed to be preserved. Thus, an unpatched LaTeX2e can allow a
% single column figure to be placed prior to an earlier double column
% figure.
% Be aware that LaTeX2e kernels dated 2015 and later have fixltx2e.sty's
% corrections already built into the system in which case a warning will
% be issued if an attempt is made to load fixltx2e.sty as it is no longer
% needed.
% The latest version and documentation can be found at:
% http://www.ctan.org/pkg/fixltx2e


%\usepackage{stfloats}
% stfloats.sty was written by Sigitas Tolusis. This package gives LaTeX2e
% the ability to do double column floats at the bottom of the page as well
% as the top. (e.g., "\begin{figure*}[!b]" is not normally possible in
% LaTeX2e). It also provides a command:
%\fnbelowfloat
% to enable the placement of footnotes below bottom floats (the standard
% LaTeX2e kernel puts them above bottom floats). This is an invasive package
% which rewrites many portions of the LaTeX2e float routines. It may not work
% with other packages that modify the LaTeX2e float routines. The latest
% version and documentation can be obtained at:
% http://www.ctan.org/pkg/stfloats
% Do not use the stfloats baselinefloat ability as the IEEE does not allow
% \baselineskip to stretch. Authors submitting work to the IEEE should note
% that the IEEE rarely uses double column equations and that authors should try
% to avoid such use. Do not be tempted to use the cuted.sty or midfloat.sty
% packages (also by Sigitas Tolusis) as the IEEE does not format its papers in
% such ways.
% Do not attempt to use stfloats with fixltx2e as they are incompatible.
% Instead, use Morten Hogholm'a dblfloatfix which combines the features
% of both fixltx2e and stfloats:
%
% \usepackage{dblfloatfix}
% The latest version can be found at:
% http://www.ctan.org/pkg/dblfloatfix




% *** PDF, URL AND HYPERLINK PACKAGES ***
%
\usepackage{url}
% url.sty was written by Donald Arseneau. It provides better support for
% handling and breaking URLs. url.sty is already installed on most LaTeX
% systems. The latest version and documentation can be obtained at:
% http://www.ctan.org/pkg/url
% Basically, \url{my_url_here}.


% *** Do not adjust lengths that control margins, column widths, etc. ***
% *** Do not use packages that alter fonts (such as pslatex).         ***
% There should be no need to do such things with IEEEtran.cls V1.6 and later.
% (Unless specifically asked to do so by the journal or conference you plan
% to submit to, of course. )


% correct bad hyphenation here
\hyphenation{op-tical net-works semi-conduc-tor}


\begin{document}
%
% paper title
% Titles are generally capitalized except for words such as a, an, and, as,
% at, but, by, for, in, nor, of, on, or, the, to and up, which are usually
% not capitalized unless they are the first or last word of the title.
% Linebreaks \\ can be used within to get better formatting as desired.
% Do not put math or special symbols in the title.
\title{Gomoku Report for Artificial Intelligence}


% author names and affiliations
% use a multiple column layout for up to three different
% affiliations
\author{\IEEEauthorblockN{Kemiao Huang  11610728}
\IEEEauthorblockA{School of Computer Science and Engineering\\
Southern University of Science and Technology\\
11610728@mail.sustc.edu.cn}}


% conference papers do not typically use \thanks and this command
% is locked out in conference mode. If really needed, such as for
% the acknowledgment of grants, issue a \IEEEoverridecommandlockouts
% after \documentclass

% for over three affiliations, or if they all won't fit within the width
% of the page (and note that there is less available width in this regard for
% compsoc conferences compared to traditional conferences), use this
% alternative format:
% 
%\author{\IEEEauthorblockN{Michael Shell\IEEEauthorrefmark{1},
%Homer Simpson\IEEEauthorrefmark{2},
%James Kirk\IEEEauthorrefmark{3}, 
%Montgomery Scott\IEEEauthorrefmark{3} and
%Eldon Tyrell\IEEEauthorrefmark{4}}
%\IEEEauthorblockA{\IEEEauthorrefmark{1}School of Electrical and Computer Engineering\\
%Georgia Institute of Technology,
%Atlanta, Georgia 30332--0250\\ Email: see http://www.michaelshell.org/contact.html}
%\IEEEauthorblockA{\IEEEauthorrefmark{2}Twentieth Century Fox, Springfield, USA\\
%Email: homer@thesimpsons.com}
%\IEEEauthorblockA{\IEEEauthorrefmark{3}Starfleet Academy, San Francisco, California 96678-2391\\
%Telephone: (800) 555--1212, Fax: (888) 555--1212}
%\IEEEauthorblockA{\IEEEauthorrefmark{4}Tyrell Inc., 123 Replicant Street, Los Angeles, California 90210--4321}}




% use for special paper notices
%\IEEEspecialpapernotice{(Invited Paper)}




% make the title area
\maketitle

% As a general rule, do not put math, special symbols or citations
% in the abstract
%\begin{abstract}

%\end{abstract}

% no keywords




% For peer review papers, you can put extra information on the cover
% page as needed:
% \ifCLASSOPTIONpeerreview
% \begin{center} \bfseries EDICS Category: 3-BBND \end{center}
% \fi
%
% For peerreview papers, this IEEEtran command inserts a page break and
% creates the second title. It will be ignored for other modes.
\IEEEpeerreviewmaketitle



\section{Preliminaries}
The goal of this project is to realize a Gomoku AI by using classical algorithms for game AI. To check how much intelligence does the AI have, the codes are submitted to the competition platform and evaluated as scores.
\subsection{Software}
This project is written by Python so Sublime Text 3 and PyCharm IDE are used since they are good for editing and debug. The libraries are numpy and functools.
\subsection{Algorithm}
The algorithms used in this project are heuristic search, negamax and alpha-beta pruning.  

The evaluation function is to compute the value which is associated with each position or state of the game. The value indicates how good it would be for a player to reach that position. The heuristic search is to pick out a number of empty positions which will most certainly win without far prediction. 

Minimax assumes that the opponent will always make the best move. It let the player make the move that maximizes the minimum value of the position resulting from the opponent's
possible following moves[1]. 

Alpha-beta pruning reduces the number of nodes that need to be evaluated in the search tree by the negamax algorithm.

Negamax algorithm relies on the fact that max(a,b) = -min(-a,-b) to simplify the implementation of the minimax algorithm. Algorithm optimizations for minimax are also equally applicable for Negamax. In this project, negamax with alpha-beta pruning is used.
% An example of a floating figure using the graphicx package.
% Note that \label must occur AFTER (or within) \caption.
% For figures, \caption should occur after the \includegraphics.
% Note that IEEEtran v1.7 and later has special internal code that
% is designed to preserve the operation of \label within \caption
% even when the captionsoff option is in effect. However, because
% of issues like this, it may be the safest practice to put all your
% \label just after \caption rather than within \caption{}.
%
% Reminder: the "draftcls" or "draftclsnofoot", not "draft", class
% option should be used if it is desired that the figures are to be
% displayed while in draft mode.
%
%\begin{figure}[!t]
%\centering
%\includegraphics[width=2.5in]{myfigure}
% where an .eps filename suffix will be assumed under latex, 
% and a .pdf suffix will be assumed for pdflatex; or what has been declared
% via \DeclareGraphicsExtensions.
%\caption{Simulation results for the network.}
%\label{fig_sim}
%\end{figure}

% Note that the IEEE typically puts floats only at the top, even when this
% results in a large percentage of a column being occupied by floats.


% An example of a double column floating figure using two subfigures.
% (The subfig.sty package must be loaded for this to work.)
% The subfigure \label commands are set within each subfloat command,
% and the \label for the overall figure must come after \caption.
% \hfil is used as a separator to get equal spacing.
% Watch out that the combined width of all the subfigures on a 
% line do not exceed the text width or a line break will occur.
%
%\begin{figure*}[!t]
%\centering
%\subfloat[Case I]{\includegraphics[width=2.5in]{box}%
%\label{fig_first_case}}
%\hfil
%\subfloat[Case II]{\includegraphics[width=2.5in]{box}%
%\label{fig_second_case}}
%\caption{Simulation results for the network.}
%\label{fig_sim}
%\end{figure*}
%
% Note that often IEEE papers with subfigures do not employ subfigure
% captions (using the optional argument to \subfloat[]), but instead will
% reference/describe all of them (a), (b), etc., within the main caption.
% Be aware that for subfig.sty to generate the (a), (b), etc., subfigure
% labels, the optional argument to \subfloat must be present. If a
% subcaption is not desired, just leave its contents blank,
% e.g., \subfloat[].


% An example of a floating table. Note that, for IEEE style tables, the
% \caption command should come BEFORE the table and, given that table
% captions serve much like titles, are usually capitalized except for words
% such as a, an, and, as, at, but, by, for, in, nor, of, on, or, the, to
% and up, which are usually not capitalized unless they are the first or
% last word of the caption. Table text will default to \footnotesize as
% the IEEE normally uses this smaller font for tables.
% The \label must come after \caption as always.
%
%\begin{table}[!t]
%% increase table row spacing, adjust to taste
%\renewcommand{\arraystretch}{1.3}
% if using array.sty, it might be a good idea to tweak the value of
% \extrarowheight as needed to properly center the text within the cells
%\caption{An Example of a Table}
%\label{table_example}
%\centering
%% Some packages, such as MDW tools, offer better commands for making tables
%% than the plain LaTeX2e tabular which is used here.
%\begin{tabular}{|c||c|}
%\hline
%One & Two\\
%\hline
%Three & Four\\
%\hline
%\end{tabular}
%\end{table}


% Note that the IEEE does not put floats in the very first column
% - or typically anywhere on the first page for that matter. Also,
% in-text middle ("here") positioning is typically not used, but it
% is allowed and encouraged for Computer Society conferences (but
% not Computer Society journals). Most IEEE journals/conferences use
% top floats exclusively. 
% Note that, LaTeX2e, unlike IEEE journals/conferences, places
% footnotes above bottom floats. This can be corrected via the
% \fnbelowfloat command of the stfloats package.




\section{Methodology}
The details of the design of the Gomoku game AI will be discussed in this section.


\subsection{Representation}
The chessboard is represented as a numpy array. In the array, 0 means empty, -1 means black and 1 means white. The pieces in chessboard can be grouped as different patterns or threat types. Different patterns are labelled as different scores.

\subsection{Architecture}
The basic parameters are defined as global variables. The AI class contains attributes of chessboard, board size, color, time out, candidate list, count of total pieces, score caches for score evaluation of self and the opponent. Most functions are defined in the class except some static functions such as compare functions.

\subsection{Details of Algorithms}
\subsubsection{Opening}
To avoid heuristic search failure, the number of total pieces in the chessboard is counted at the beginning of the 'go' function. To be clear, a simple opening function is written.  

\begin{algorithm}
 \caption{simple opening}
 \begin{algorithmic}[h]
 \renewcommand{\algorithmicrequire}{\textbf{Input:}}
 \renewcommand{\algorithmicensure}{\textbf{Output:}}
 \REQUIRE $chessboard$
 \ENSURE  position of next move\\ 
  \IF {$chessboard$ is None}
  \RETURN None
  \ENDIF
  \STATE $count$ = number of pieces in $chessboard$ \\
  \IF {($count > 1$)}
  \RETURN None
  \ENDIF \\
  \IF {($count == 0$)}
  \STATE index = center of $chessboard$
  \RETURN $index$
  \ENDIF \\
  \RETURN index next to the first piece at the opposite side respect to the chessboard
 \end{algorithmic} 
 \end{algorithm}
\subsubsection{Score Evaluation}
After the opening, the scores of the empty places are initialized for heuristic search. To reduce the time, all the empty places whose close neighbours are less than 2 are discarded when doing evaluation. The \textit{evaluation} function is to evaluate the score for one index.

\begin{algorithm}
 \caption{evaluation}
 \begin{algorithmic}[h]
 \renewcommand{\algorithmicrequire}{\textbf{Input:}}
 \renewcommand{\algorithmicensure}{\textbf{Output:}}
 \REQUIRE $piece\_index$, $color$, $direction$
 \ENSURE score of one empty space 
  \FOR {$dir$ in the 4 directions}
  \IF {$direction$ is None or $dir$ == $direction$}
  \STATE $cnt_1$ = $cnt_2$ = 1, $empty\_loc$ = 0, $end_1$ = $end_2$ = 0, $empty_2$ = False
  \STATE do the first half traverse for $cnt_1$, $end_1$, $empty_2$, $empty\_loc$
  \IF {$empty_2$ is not None}
  \STATE do the second half traverse for $cnt_2$, $end_2$, $empty_2$, $empty\_loc$
  \ENDIF
  \STATE use the values above to match the patterns and store the score in $cache$
  \ENDIF
  \ENDFOR
  \RETURN sum of scores at index in $cache$
 \end{algorithmic} 
 \end{algorithm}
 
Generally, the evaluation for each direction is divided into two parts from the piece we focus. I use this kind of approach because it helps to check the pattern when there is a more than five loose connected with one empty space in it. Simply, if that situation occurs, it is either a connected five or a live four or a flush four. I can just check each part and the empty space location to get the pattern straightforward.

To reduce the time cost, the parameter 'direction' is used. The reason is that each time adding or removing a piece from the chessboard, the scores of spaces around the piece should be updated. The score of each piece is only changed through the direction of the line connected by itself and the new added or removed piece. On the other directions, it doesn't need to update.

Additionally, the $empty_2$ variable is used for check 'big jump live two', which is a pattern equivalent to live two and jump live two. 

\subsubsection{Heuristic Search}
Heuristic function is to group up the empty pieces by their score levels and return the pieces list with best score level. Obviously, it needs some strategies to group the pieces to control the balance of keeping the number of the output small enough and ensuring the output contains the piece which will have the biggest threat. 

\begin{algorithm}
 \caption{heuristic search}
 \begin{algorithmic}[h]
 \renewcommand{\algorithmicrequire}{\textbf{Input:}}
 \renewcommand{\algorithmicensure}{\textbf{Output:}}
 \REQUIRE $color$
 \ENSURE  pieces list with biggest threat\\
 \STATE initialize the empty lists: $five$, $live\_four$, $flush\_four$, $two\_three$, $three$, $two$, $one$ for self and opponent respectively.
 \FOR{empty $piece$ in chessboard}
 \STATE discard the too sparse pieces 
 \STATE get the scores from caches for self and opponent
 \STATE append $piece$ to one right pattern list
 \ENDFOR
 \STATE sort the lists
 \STATE check each list whether it is empty for self first and then opponent
 \IF{$five$ or $live\_four$ is not empty}
 \RETURN $five$ or $live\_four$
 \ENDIF
 \STATE $result$ = combine the left non-empty lists in order by different priorities
 \IF {$len(result)>max\_len$}
 \STATE cut the tail of $result$ 
 \ENDIF
 \RETURN $result$
 \end{algorithmic} 
 \end{algorithm}

\subsubsection{Negamax and Alpha-Beta Pruning}
Negamax algorithm is executed to deepen the search. Although the alpha-beta pruning is used, the time complexity is still very large compared to the other functions in the game. Actually, the performance of using negamax is greatly depends on the evaluation function.  

\begin{algorithm}
 \caption{negemax with alpha-beta pruning}
 \begin{algorithmic}[h]
 \renewcommand{\algorithmicrequire}{\textbf{Input:}}
 \renewcommand{\algorithmicensure}{\textbf{Output:}}
 \REQUIRE $self$, $depth$, $alpha$, $beta$, $color$
 \ENSURE  best score with step\\
  \STATE $pieces\_list$ = pieces given by heuristic search
  \IF {$depth$ == $0$ or $pieces\_list$ is empty or $five$ score piece $\in$ $pieces\_list$}
  \STATE$self\_max$ = $rival\_max$ = $0$\\ 
  \FOR {$piece$ in empty places}
  \STATE $self\_max$ = max($self\_max$, $piece$ score in self score cache)
  \STATE $rival\_max$ = max($rival\_max$, $piece$ score in rival score cache)
  \ENDFOR
  \IF {$color$ == $self.color$}
  \RETURN $self\_max - rival\_max$
  \ELSE
  \RETURN $rival\_max - self\_max$
  \ENDIF
  \ENDIF  
  \STATE $best$ = -$\infty$
  \FOR {$piece$ in $pieces\_list$}
  \STATE put $piece$
  \STATE $v$ = -$alphabeta$($depth-1$, $-beta$, $-alpha$, $-color$)
  \IF {$v>best$}
  \STATE $best$ = $v$
  \ENDIF
  \STATE $alpha$ = max($alpha$, $v.score$)
  \STATE remove $piece$
  \IF {$v$ \textit{greater than} $beta$}
  \RETURN $beta$
  \ENDIF
  \ENDFOR
 \RETURN $best$ 
 \end{algorithmic} 
 \end{algorithm}

\subsubsection{Sort and Comparison}
After adjusting the scores by negamax algorithm, The candidate list should be sorted to get the best move. The sort() function in python can be used but the comparison function in sort in not allowed in Python 3. Therefore, I use the library 'functools' to convert the comparison function to key.

\begin{algorithm}
 \caption{compare scores}
 \begin{algorithmic}[h]
 \renewcommand{\algorithmicrequire}{\textbf{Input:}}
 \renewcommand{\algorithmicensure}{\textbf{Output:}}
 \REQUIRE $a$, $b$
 \ENSURE comparison value 
  \IF {$a.max\_score$ \textit{equal} $b.max\_score$}
  \RETURN $b.totalScore$ - $a.totalScore$
  \ENDIF
  \RETURN $b.maxScore$ - $a.maxScore$
\end{algorithmic} 
\end{algorithm}

Moreover, the equal, greater, less et.al functions are given to approximately compare the value instead of using absolute comparison. The goal is to reserve the approximate results for further comparisons in algorithms of negamax and sorting. This method is referred from [2].

\begin{algorithm}
 \caption{equal}
 \begin{algorithmic}[h]
 \renewcommand{\algorithmicrequire}{\textbf{Input:}}
 \renewcommand{\algorithmicensure}{\textbf{Output:}}
 \REQUIRE $a$, $b$
 \ENSURE comparison value 
  \STATE $b$ = $b$ or $0.01$
  \IF {$b\geq0$}
  \RETURN $b/threshold\leq a\leq b*threshold$
  \ENDIF
  \RETURN $b*threshold\leq a\leq b/threshold$
 \end{algorithmic} 
 \end{algorithm}
 
\begin{algorithm}
 \caption{greater}
 \begin{algorithmic}[h]
 \renewcommand{\algorithmicrequire}{\textbf{Input:}}
 \renewcommand{\algorithmicensure}{\textbf{Output:}}
 \REQUIRE $a$, $b$
 \ENSURE comparison value 
  \IF {$b\geq0$}
  \RETURN $a\geq (b+0.1)*threshold$
  \ENDIF
  \RETURN $a\geq (b+0.1)/threshold$
 \end{algorithmic} 
 \end{algorithm}
 

The two big difficulties in the algorithm implementation are the pattern matching and using recursion in pruning function. The number of code lines for matching and score evaluation is about 500 and pruning recursion is difficult to debug since the moves of pieces are too many.   

 
\section{Empirical Verification}
This section will discuss the test and result part of this project.

\subsection{Design}
Although the preliminary test file is provided, the test data is not enough for performance guarantee. Followed by the provided 'code\_check.py' file, modify the '\_check\_advance\_chessboard()' function and the expected result list can test all the specific chessboard as desired.
Fortunately, the school has built a perfect on-line competition platform for us to upload the code and play against each other. The chess logs are reserved to modify the parameters the strategy for the algorithms, especially the evaluation function and the constant scores for different patterns.

\subsection{Data}
According to experience, the basic test should be able to check whether it can defence the two-live-three or flush-four-live-three threats. The number of the good way to attack is usually not only one so the test actually can only focus on defence.  
\subsection{Performance}
The speed of negamax is the bottleneck of the entire program. The required time for one move is less than five seconds. In this project, interrupt function is not set so to ensure the result comes before time out the search depth for negamax is set as two.
\subsection{Result}
The evaluation of different patterns are summarized as a table. The self part should be larger than the opponent because attack has higher priority than defence. The scores are mostly set by experience. The learning algorithm is not considered in this project.

\begin{table}[ht]
\caption{Scores for Chess Patterns}
\centering
\begin{tabular}{ccc}
\hline
\textbf{pattern}   & \textbf{self} & \textbf{opponent} \\ \hline
sleep one          & 15            & 10                \\
live one           & 40            & 25                \\
sleep two          & 120           & 75                \\
(big)jump live two & 260           & 240               \\
live two           & 450           & 415               \\
sleep three        & 650           & 500               \\
jump live three    & 1550          & 1150              \\
live three         & 1730          & 1425              \\
flush four         & 2450          & 1750              \\
live four          & 4750          & 3600              \\
five               & 20000         & 15000             \\ \hline
\end{tabular}
\end{table}

The other parameters in the program are all by testing. 

\begin{table}[ht]
\caption{Other Parameters}
\centering
\begin{tabular}{cc}
\hline
\textbf{parameter} & \textbf{value} \\ \hline
search depth       & 2              \\
threshold          & 1.4            \\
heuristic max len  & 10             \\ \hline
\end{tabular}
\end{table}

The speed of each move with negamax algorithm is about 2 seconds. If the search depth is set as zero, the time cost is usually less than 0.5 second.


\subsection{Analysis}
Five seconds are usually sufficient for one move. However, when the search depth is set as 4 or larger, the result sometimes goes wrong and the AI is not much smarter. It is inferred that there may be some bugs when implementing the negamax algorithm. 
To further the searching algorithm, some strategies such as continuously finding moves for flush four and finding the check pieces for two-threes and flush-four-live-three can be used. Actually, those strategies can be ignored if the search depth is large enough but the benefit for those strategies is to save some time when doing search. 


% trigger a \newpage just before the given reference
% number - used to balance the columns on the last page
% adjust value as needed - may need to be readjusted if
% the document is modified later
%\IEEEtriggeratref{8}
% The "triggered" command can be changed if desired:
%\IEEEtriggercmd{\enlargethispage{-5in}}

% references section

% can use a bibliography generated by BibTeX as a .bbl file
% BibTeX documentation can be easily obtained at:
% http://mirror.ctan.org/biblio/bibtex/contrib/doc/
% The IEEEtran BibTeX style support page is at:
% http://www.michaelshell.org/tex/ieeetran/bibtex/
\bibliographystyle{IEEEtran}
% argument is your BibTeX string definitions and bibliography database(s)
%\bibliography{IEEEabrv,../bib/paper}
%
% <OR> manually copy in the resultant .bbl file
% set second argument of \begin to the number of references
% (used to reserve space for the reference number labels box)
\begin{thebibliography}{1}
\bibitem{reference}
Kuan Liang Tan, C. H. Tan, K. C. Tan and A. Tay, "Adaptive game AI for Gomoku," 2009 4th International Conference on Autonomous Robots and Agents, Wellington, 2000, pp. 507-512.

\bibitem{reference}
Li, H. (2018). lihongxun945/gobang. [online] GitHub. Available at: https://github.com/lihongxun945/gobang [Accessed 12 Oct. 2018].
\end{thebibliography}





% that's all folks
\end{document}


